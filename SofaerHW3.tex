\documentclass[11pt]{modart}
\usepackage{geometry}                % See geometry.pdf to learn the layout options. There are lots.
\geometry{letterpaper}                   % ... or a4paper or a5paper or ... 
%\geometry{landscape}                % Activate for for rotated page geometry
%\usepackage[parfill]{parskip}    % Activate to begin paragraphs with an empty line rather than an indent
\usepackage{graphicx}
\usepackage{amssymb}
\usepackage{lastpage}
\usepackage{epstopdf}
\usepackage{fancyhdr}
\DeclareGraphicsRule{.tif}{png}{.png}{`convert #1 `dirname #1`/`basename #1 .tif`.png}

% Homework Specific Information
\newcommand{\hmwkTitle}{Homework 3}
\newcommand{\hmwkDueDate}{February 21, 2012}
\newcommand{\hmwkClass}{Numerical Computing}
\newcommand{\hmwkClassInstructor}{Margaret Wright}
\newcommand{\hmwkAuthorName}{Raphael Sofaer}

% Setup the header and footer
\pagestyle{fancy}                                                       %
\lhead{\hmwkAuthorName}                                                 %
\rhead{\hmwkClass\ (\hmwkClassInstructor): \hmwkTitle}  %
\lfoot{\lastxmark}                                                      %
\cfoot{}                                                                %
\rfoot{Page\ \thepage\ of\ \pageref{LastPage}}                          %
\renewcommand\headrulewidth{0.4pt}                                      %
\renewcommand\footrulewidth{0.4pt}                                      %

\title{\large{\hmwkAuthorName}\vspace{0.1in}\\\textmd{\textbf{\hmwkClass:\ \hmwkTitle}}\\\normalsize\vspace{0.1in}\small{Due\ on\ \hmwkDueDate}\\\vspace{0.1in}\large{\textit{\hmwkClassInstructor}}\vspace{0.5in}}
\author{}
\date{}
  
\begin{document}
\maketitle

\section{Exercise 3.1}
\subsection{Compute $y= x^0$ when:}
a. $0^0=1$.  In hex:\\
$0000000000000000^{0000000000000000} = 3ff0000000000000$, which makes sense, since $x^0 = 1$.\\
\newline
b. $inf^0 = 1$. In hex:\\
$7ff0000000000000^{0000000000000000} = 3ff0000000000000$, which makes sense by the same logic.\\
\newline
c. $NaN^0 = 1$. In hex:\\
$7ff8000000000000^{0000000000000000} = 3ff0000000000000$, which is the same.\\
\newline
Conceptually, this rule makes sense to me if I think of $x^y$ as $1*x_1*x_2*\cdots*x_{y}$.  If $y=0$, $x^y=1$.
\subsection{Do the same for:}
a. $1^{Inf}=1$.  In hex:\\
$3ff0000000000000^{7ff0000000000000} = 3ff0000000000000$,
 which makes sense, since no number of 1s multiplied together can give anything but 1.\\

b. $-1^{Inf}=NaN - NaNi$.  In hex:\\
$bff0000000000000^{7ff0000000000000} = fff8000000000000  fff8000000000000i$. 
 This seems arbitrary and a little strange to me, since $-1^n$ is one of {1,-1}.\\

c. $log(0.0)=-Inf$.  In hex:\\
$log(0000000000000000)=fff0000000000000$.  This makes sense, since the limit of log(x) as x goes to 0 is -infinity.\\

d. $log(-Inf)=Inf + 3.132i$.  In hex:\\
$log(fff0000000000000)=7ff0000000000000  400921fb54442d18i$.
This makes sense because if $log(x)=y$, $e^y = x$, and since by Euler's identity $$e^{\pi i}=-1$$ 
$$log(-1)=\pi i$$
$$log(-Inf)=log(-1) + log(Inf)$$
$$log(Inf)=Inf$$
$$log(Inf) + log(-1) = Inf + \pi i$$

e. $exp(-Inf) = 0$.  In hex:\\
$exp(fff0000000000000) = 0000000000000000$.  This makes sense because $e^{-Inf} = 1/e^{Inf} = 1/Inf = 0$.

\subsection{A non-standard calculation of my own:}
I tried $Inf/Inf$, which is NaN.  That makes sense to me, because a value of 1 would imply that Inf is a number, but it's more of a limit or representation of a concept than a number.

\section{Hexadecimal IEEE numbers in decimal}
\subsection{4059000000000000}
The sign bit and exponent of this number is 405 in hex, which is 0100 0000 0101.  Without the sign bit, the exponent bitstring is 100 0000 0101, or still 405 in hex, or $4*16^2 + 5$, which comes out to 1029.  The exponent is 1029-1023, or 6.\\
The mantissa of this number is 9000000000000, or 1001 followed by 48 zeros in binary.  When the implied 1 is included, we get 1.1001 followed by 48 zeros.\\
In decimal, we have $(1 + 1/2 + 1/16) * 2^6=1.5625*2^6=100$.
\subsection{3f847ae147ae147b}
The sign bit is 0, and the exponent bitstring is 3f8, so the exponent is $-1023 + 3*16^2 + 15*16 + 8=1016-1023 = -7$.\\
That leaves a mantissa of 1.47ae147ae147b. \\
In decimal, this is $(1 + 4/16 + 7/16^2 + ... + 11/16^{13})*2^{-7}$.  I used a ruby interpreter to calculate this,
and it came out to $1.28*2^{-7} = 0.01$.
\subsection{3fe921fb54442d18}
The sign bit is 0, and the exponent bitstring is 3fe, which gives an exponent of $-1023 + 3*16^2 + 15*16 + 14 = -1$.\\
That leaves a mantissa of 1.921fb54442d18.\\
In decimal, I used the ruby interpreter again to come out with $1.5707963267948966/2=\pi/4$.
\section{Error analysis of the Taylor series expansion of a smooth function f}
\subsection{Give an upper bound to the truncation error $|e_\tau|$ of (1.2) expressed in terms of the finite-difference interval h, that is valid for all $\bar{x}$.}

\section{Consider $Ax=b$, where A is a nonsingular $n$ x $n$ matrix with $n>1$.}
\subsection{Solve the given linear system}
By inspection, the solution given linear system is $x^*=[1, -1]$.\\
In hex, this is $x^*=[3ff0000000000000, bff0000000000000]$\\
Matlab gives the answer $\tilde{x}=[1.00000000000020, -1.00000000000027]$.\\
In hex, this is $\tilde{x} = [3ff000000000039b, bff00000000004b0]$.\\
$$\tilde{x}-x^*= [-2.04947170345804 * 10^{-13}, 2.66453525910038 * 10^{-13}]$$
In hex, this is $\tilde{x}-x^* = [bd4cd80000000000 3d52c00000000000]$.\\
\subsection{Compute the residuals $r*$ and $\tilde{r}$}

\end{document}
