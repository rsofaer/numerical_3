\documentclass[11pt]{modart}
\usepackage{geometry}                % See geometry.pdf to learn the layout options. There are lots.
\geometry{letterpaper}                   % ... or a4paper or a5paper or ... 
%\geometry{landscape}                % Activate for for rotated page geometry
%\usepackage[parfill]{parskip}    % Activate to begin paragraphs with an empty line rather than an indent
\usepackage{graphicx}
\usepackage{amssymb}
\usepackage{lastpage}
\usepackage{epstopdf}
\usepackage{fancyhdr}
\DeclareGraphicsRule{.tif}{png}{.png}{`convert #1 `dirname #1`/`basename #1 .tif`.png}

% Homework Specific Information
\newcommand{\hmwkTitle}{Homework 3}
\newcommand{\hmwkDueDate}{February 21, 2012}
\newcommand{\hmwkClass}{Numerical Computing}
\newcommand{\hmwkClassInstructor}{Margaret Wright}
\newcommand{\hmwkAuthorName}{Raphael Sofaer}

% Setup the header and footer
\pagestyle{fancy}                                                       %
\lhead{\hmwkAuthorName}                                                 %
\rhead{\hmwkClass\ (\hmwkClassInstructor): \hmwkTitle}  %
\lfoot{\lastxmark}                                                      %
\cfoot{}                                                                %
\rfoot{Page\ \thepage\ of\ \pageref{LastPage}}                          %
\renewcommand\headrulewidth{0.4pt}                                      %
\renewcommand\footrulewidth{0.4pt}                                      %

\title{\large{\hmwkAuthorName}\vspace{0.1in}\\\textmd{\textbf{\hmwkClass:\ \hmwkTitle}}\\\normalsize\vspace{0.1in}\small{Due\ on\ \hmwkDueDate}\\\vspace{0.1in}\large{\textit{\hmwkClassInstructor}}\vspace{0.5in}}
\author{}
\date{}
  
\begin{document}
\maketitle

\section{Exercise 3.1}
\subsection{Compute $y= x^0$ when:}
a. $0^0=1$.  In hex:\\
$0000000000000000^{0000000000000000} = 3ff0000000000000$, which makes sense, since $x^0 = 1$.\\
b. $inf^0 = 1$. In hex:\\
$7ff0000000000000^{0000000000000000} = 3ff0000000000000$, which makes sense by the same logic.\\
c. $NaN^0 = 1$. In hex:\\
$7ff8000000000000^{0000000000000000} = 3ff0000000000000$, which is the same.\\
Conceptually, this rule makes sense to me if I think of $x^y$ as $1*x_1*x_2*\cdots*x_{y}$.  If $y=0$, $x^y=1$.
\subsection{Do the same for:}
a. $1^{Inf}=1$.  In hex:\\
$3ff0000000000000^{7ff0000000000000} = 3ff0000000000000$, which makes sense, since no number of 1s multiplied together can give anything but 1.\\

b. $-1^{Inf}=NaN - NaNi$.  In hex:\\
$bff0000000000000^{7ff0000000000000} = fff8000000000000  fff8000000000000i$.  This seems arbitrary and a little strange to me, since $-1^n$ is one of {1,-1}.

\end{document}
